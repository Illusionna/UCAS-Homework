\documentclass{article}

\usepackage{bm}
\usepackage{float}
\usepackage{xcolor}
\usepackage{pifont}
\usepackage{booktabs}
\usepackage{setspace}
\usepackage{datetime}
\usepackage{geometry}
\usepackage{graphicx}
\usepackage{mathrsfs}
\usepackage{listings}
\usepackage{tcolorbox}
\usepackage{amsmath, amssymb, amsfonts}
\usepackage[fontsize = 12pt]{fontsize}

\renewcommand{\thesection}{\Roman{section}}
\renewcommand{\thesubsection}{\arabic{subsection}}
\renewcommand{\thesubsubsection}{(\alph{subsubsection})}
\newcommand{\diff}[2]{\dfrac{\mathrm{d}#1}{\mathrm{d}#2}}
\newcommand{\diffn}[3]{\dfrac{\mathrm{d}^{#3}#1}{\mathrm{d}#2^{#3}}}
\newcommand{\piff}[2]{\dfrac{\partial#1}{\partial#2}}
\newcommand{\piffn}[3]{\dfrac{\partial^{#3}#1}{\partial#2^{#3}}}
\newcommand{\eval}[2]{\left.#1\right|_{#2}}
\newcommand{\e}{\mathrm{e}}
\renewcommand{\d}{\mathrm{d}}

\definecolor{gray}{RGB}{128, 128, 128}

\geometry{
    a4paper,
    left = 2cm,
    right = 2cm,
    top = 2cm,
    bottom = 2cm
}

\title{Data Mining: Homework 2}
\author{Illusionna \quad 2025.............4 \quad Artificial Intelligence School}
\date{\currenttime,\ \today}



\begin{document}

\maketitle
\begin{spacing}{1.5}
    \tableofcontents
\end{spacing}
\thispagestyle{empty}
\clearpage
\setcounter{page}{1}


\section{Written Assignment}
\subsection{ID3 Algorithm}
\begin{table}[H]
    \centering
    \renewcommand{\arraystretch}{1}
    \setlength{\tabcolsep}{12pt}
    \color{gray}{
        \caption{The customer data overview (Q1)}
        \begin{tabular}{|c|c|c|c|c|}
            \hline
            Customer ID & Gender & Car Type & Shirt Size & Class \\
            \hline
            1  & M & Family & Small & C0 \\
            2  & M & Sports & Medium & C0 \\
            3  & M & Sports & Medium & C0 \\
            4  & M & Sports & Large & C0 \\
            5  & M & Sports & Extra Large & C0 \\
            6  & M & Sports & Extra Large & C0 \\
            7  & F & Sports & Small & C0 \\
            8  & F & Sports & Small & C0 \\
            9  & F & Sports & Medium & C0 \\
            10 & F & Luxury & Large & C0 \\
            11 & M & Family & Large & C1 \\
            12 & M & Family & Extra Large & C1 \\
            13 & M & Family & Medium & C1 \\
            14 & M & Luxury & Extra Large & C1 \\
            15 & F & Luxury & Small & C1 \\
            16 & F & Luxury & Small & C1 \\
            17 & F & Luxury & Medium & C1 \\
            18 & F & Luxury & Medium & C1 \\
            19 & F & Luxury & Medium & C1 \\
            20 & F & Luxury & Large & C1 \\
            \hline
        \end{tabular}
    }
\end{table}



\subsubsection{Compute the information gain}
\textcolor{gray}{\it Compute the Information Gain for Gender, Car Type and Shirt Size.}

\[ IG(S)={\rm Entropy}(D)-\sum_{n\in S}\dfrac{\vert D_n\vert}{\vert D\vert}\cdot{\rm Entropy}(D_n) \]

\[ {\rm Entropy}(X)=-\sum_{n=1}^Np_n\log_2p_n \]

Among: $D$ denotes the class of dataset, $D_n$ denotes a certain attribute of sub-dataset. $|\sim|$ denotes the element count.

\[ {\rm C0=C1=10}\implies p({\rm C0})=p({\rm C1})=0.5\implies {\rm Entropy}({\rm Class})=-0.5\log_20.5\times2=1 \]

\begin{table}[H]
    \centering
    \renewcommand{\arraystretch}{1.2}
    \setlength{\tabcolsep}{36pt}
    \caption{Gender Attribute}
    \begin{tabular}{cccc}
        \bottomrule[1.5pt]
        Gender & C0 & C1 & Entropy \\
        \midrule[0.75pt]
        M & 6 & 4 & 0.9709505944546686 \\
        F & 4 & 6 & 0.9709505944546686 \\
        \toprule[1.5pt]
    \end{tabular}
\end{table}

\[ IG({\rm Gender})=1-\dfrac{6+4}{20}\times0.9709505944546686\times2=\color{red}0.02904940554533142 \]

\begin{table}[H]
    \centering
    \renewcommand{\arraystretch}{1.2}
    \setlength{\tabcolsep}{36pt}
    \caption{Car Type Attribute}
    \begin{tabular}{cccc}
        \bottomrule[1.5pt]
        Gender & C0 & C1 & Entropy \\
        \midrule[0.75pt]
        Family & 1 & 3 & 0.8112781244591328 \\
        Sports & 8 & 0 & 0 \\
        Luxury & 1 & 7 & 0.5435644431995964 \\
        \toprule[1.5pt]
    \end{tabular}
\end{table}

\[ IG({\rm Car\ Type})=1-\dfrac{4}{20}\times0.8112781244591328-\dfrac{8}{20}\times0.5435644431995964=\color{red}0.620318597828335 \]

\begin{table}[H]
    \centering
    \renewcommand{\arraystretch}{1.2}
    \setlength{\tabcolsep}{5pt}
    \caption{Shirt Size Attribute}
    \begin{tabular}{cccc}
        \bottomrule[1.5pt]
        Shirt Size & C0 & C1 & Entropy \\
        \midrule[0.75pt]
        Small & 3 & 2 & $-\dfrac{3}{3+2}\log_2\left(\dfrac{3}{3+2}\right)-\dfrac{2}{3+2}\log_2\left(\dfrac{2}{3+2}\right)\approx0.9709505944546686$ \\
        Medium & 3 & 4 & $-\dfrac{3}{3+4}\log_2\left(\dfrac{3}{3+4}\right)-\dfrac{4}{3+4}\log_2\left(\dfrac{4}{3+4}\right)\approx0.9852281360342515$ \\
        Large & 2 & 2 & $-\dfrac{2}{2+2}\log_2\left(\dfrac{2}{2+2}\right)-\dfrac{2}{2+2}\log_2\left(\dfrac{2}{2+2}\right)=1$ \\
        Extra Large & 2 & 2 & $-\dfrac{2}{2+2}\log_2\left(\dfrac{2}{2+2}\right)-\dfrac{2}{2+2}\log_2\left(\dfrac{2}{2+2}\right)=1$ \\
        \toprule[1.5pt]
    \end{tabular}
\end{table}

\begin{align*}
    IG({\rm Shirt\ Size})&={\rm Entropy}({\rm Class})-\sum_{n\in{\rm Shirt\ Size}}\dfrac{\vert D_n\vert}{\vert\rm Class\vert}\cdot{\rm Entropy}(D_n)
    \\
    &=E({\rm Class})-\dfrac{\vert\rm Small\vert}{\vert\rm Class\vert}E({\rm Small})-\dfrac{\vert\rm Medium\vert}{\vert\rm Class\vert}E({\rm Medium})
    \\
    &\quad-\dfrac{\vert\rm Large\vert}{\vert\rm Class\vert}E({\rm Large})-\dfrac{\vert\rm Extra\ Large\vert}{\vert\rm Class\vert}E({\rm Extra\ Large})
    \\
    &=1-\dfrac{3+2}{20}\times0.9709505944546686-\dfrac{3+4}{20}\times0.9852281360342515
    \\
    &\quad-\dfrac{2+2}{20}\times1-\dfrac{2+2}{20}\times1
    \\
    &=\color{red}0.012432503774344739
\end{align*}



\subsubsection{Construct a decision tree}
\textcolor{gray}{\it Construct a decision tree with Information Gain.}
\paragraph*{Step 1}~{}

Use ``Car Type'' as the selection attribute to partition the current dataset.

\begin{figure}[H]
    \centering
    \includegraphics*[scale=0.45]{./figs/step1.pdf}
    \caption{Decision tree step 1}
\end{figure}

\paragraph*{Step 2}~{}

\textbf{For the dataset partitioned by ``Car Type = Family'', we calculate the information gain in this subset as follows:}

\begin{table}[H]
    \centering
    \renewcommand{\arraystretch}{1}
    \setlength{\tabcolsep}{12pt}
    \caption{Car Type = Family}
    \begin{tabular}{|c|c|c|c|c|}
        \hline
        Customer ID & Gender & Car Type & Shirt Size & Class \\
        \hline
        1  & M & Family & Small & C0 \\
        11 & M & Family & Large & C1 \\
        12 & M & Family & Extra Large & C1 \\
        13 & M & Family & Medium & C1 \\
        \hline
    \end{tabular}
\end{table}

\[ {\rm Entropy}({\rm Class})=-\dfrac{1}{4}\log_2\left(\dfrac{1}{4}\right)-\dfrac{3}{4}\log_2\left(\dfrac{3}{4}\right)\approx0.8112781244591328 \]

\begin{table}[H]
    \centering
    \renewcommand{\arraystretch}{1.2}
    \setlength{\tabcolsep}{36pt}
    \caption{Gender Attribute (Car Type = Family)}
    \begin{tabular}{cccc}
        \bottomrule[1.5pt]
        Gender & C0 & C1 & Entropy \\
        \midrule[0.75pt]
        M & 1 & 3 & 0.8112781244591328 \\
        F & 0 & 0 & 0 \\
        \toprule[1.5pt]
    \end{tabular}
\end{table}

\[ IG({\rm Gender})=0.8112781244591328-\dfrac{1+3}{4}\times(0.8112781244591328)=\color{red}0 \]

\begin{table}[H]
    \centering
    \renewcommand{\arraystretch}{1.2}
    \setlength{\tabcolsep}{40pt}
    \caption{Shirt Size Attribute (Car Type = Family)}
    \begin{tabular}{cccc}
        \bottomrule[1.5pt]
        Shirt Size & C0 & C1 & Entropy \\
        \midrule[0.75pt]
        Small & 1 & 0 & 0 \\
        Medium & 0 & 1 & 0 \\
        Large & 0 & 1 & 0 \\
        Extra Large & 0 & 1 & 0 \\
        \toprule[1.5pt]
    \end{tabular}
\end{table}

\[ IG({\rm Shirt\ Size})=0.8112781244591328-\dfrac{1+0}{4}\times0-\dfrac{0+1}{4}\times0\times3=\color{red}0.8112781244591328 \]
\[ IG({\rm Shirt\ Size})>IG({\rm Gender}) \]

Select ``Shirt Size'' as partition attribute. \textcolor{red}{Note that all information entropy in ``Shirt Size'' is 0 and no branching is required.}

\textbf{For the dataset partitioned by ``Car Type = Luxury'', we calculate the information gain in this subset as follows:}

\begin{table}[H]
    \centering
    \renewcommand{\arraystretch}{1}
    \setlength{\tabcolsep}{12pt}
    \caption{Car Type = Luxury}
    \begin{tabular}{|c|c|c|c|c|}
        \hline
        Customer ID & Gender & Car Type & Shirt Size & Class \\
        \hline
        10 & F & Luxury & Large & C0 \\
        14 & M & Luxury & Extra Large & C1 \\
        15 & F & Luxury & Small & C1 \\
        16 & F & Luxury & Small & C1 \\
        17 & F & Luxury & Medium & C1 \\
        18 & F & Luxury & Medium & C1 \\
        19 & F & Luxury & Medium & C1 \\
        20 & F & Luxury & Large & C1 \\
        \hline
    \end{tabular}
\end{table}

\[ {\rm Entropy}({\rm Class})=-\dfrac{1}{8}\log_2\left(\dfrac{1}{8}\right)-\dfrac{7}{8}\log_2\left(\dfrac{7}{8}\right)\approx0.5435644431995964 \]

\begin{table}[H]
    \centering
    \renewcommand{\arraystretch}{1.2}
    \setlength{\tabcolsep}{36pt}
    \caption{Gender Attribute (Car Type = Luxury)}
    \begin{tabular}{cccc}
        \bottomrule[1.5pt]
        Gender & C0 & C1 & Entropy \\
        \midrule[0.75pt]
        M & 0 & 1 & 0 \\
        F & 1 & 6 & 0.5916727785823275 \\
        \toprule[1.5pt]
    \end{tabular}
\end{table}

\begin{align*}
    IG({\rm Gender})&=0.5435644431995964-\dfrac{0+1}{8}\times0-\dfrac{1+6}{8}\times0.5916727785823275
    \\
    &\approx\color{red}0.025850761940059863
\end{align*}

\begin{table}[H]
    \centering
    \renewcommand{\arraystretch}{1.2}
    \setlength{\tabcolsep}{40pt}
    \caption{Shirt Size Attribute (Car Type = Luxury)}
    \begin{tabular}{cccc}
        \bottomrule[1.5pt]
        Shirt Size & C0 & C1 & Entropy \\
        \midrule[0.75pt]
        Small & 0 & 2 & 0 \\
        Medium & 0 & 3 & 0 \\
        Large & 1 & 1 & $\color{green}1^{\dagger}$ \\
        Extra Large & 0 & 1 & 0 \\
        \toprule[1.5pt]
    \end{tabular}
\end{table}

\begin{align*}
    IG({\rm Shirt\ Size})&=0.5435644431995964-\dfrac{0+2}{8}\times0-\dfrac{0+3}{8}\times0-\dfrac{1+1}{8}\times1-\dfrac{0+1}{8}\times0
    \\
    &=\color{red}0.2935644431995964
\end{align*}

\[ IG({\rm Shirt\ Size})>IG({\rm Gender}) \]

Select ``Shirt Size'' as partition attribute. \textcolor{red}{Note that ``Large'' information entropy in ``Shirt Size'' is $\color{green}1^{\dagger}$ and branching is still required.}

\begin{figure}[H]
    \centering
    \includegraphics*[scale=0.7]{./figs/step2.pdf}
    \caption{Decision tree step 2}
\end{figure}

\paragraph*{Step 3}~{}

\textbf{For the dataset partitioned by ``Car Type = Family'' \& ``Shirt Size = Large'', the ``Gender'' attribute is only available here:}

\begin{table}[H]
    \centering
    \renewcommand{\arraystretch}{1}
    \setlength{\tabcolsep}{12pt}
    \caption{Car Type = Luxury \& Shirt Size = Large}
    \begin{tabular}{|c|c|c|c|c|}
        \hline
        Customer ID & Gender & Car Type & Shirt Size & Class \\
        \hline
        10 & F & Luxury & Large & C0 \\
        20 & F & Luxury & Large & C1 \\
        \hline
    \end{tabular}
\end{table}

Since in the ``Gender'' attribute, one belongs to C0 and the other to C1, it is a tie. The phenomenon of branch conflicts caused by data inconsistency has emerged. The \textcolor{red}{majority voting strategy} is applied to determine the node. It is important to trace back to the parent node. The prior probability of parent node (Car Type = Luxury) is:
\[ {\rm C0=1\quad C1=7} \]

Because C1 in parent node has an absolute advantage, the key of tie-breaking is \textcolor{red}{choosing C1 when the child node} (Car Type = Luxury \& Shirt Size = Large) is tie.

After pruning, the final decision tree is given:

\begin{figure}[H]
    \centering
    \includegraphics*[scale=0.7]{./figs/step3.pdf}
    \caption{Decision tree step 3}
\end{figure}



\subsection{Neural Network}
\subsubsection{Design a multilayer feed-forward neural network}
\textcolor{gray}{\it Design a multilayer feed-forward neural network (one hidden layer) for the data set in Q1 (customer data overview). Label the nodes in the input and output layers.}

\begin{figure}[H]
    \centering
    \includegraphics*[scale=0.35]{./figs/neural-network.pdf}
    \caption{Multilayer feed-forward neural network}
\end{figure}
\vspace*{-2em}
\begin{align*}
    {\rm Gender}\begin{cases}
        {\rm M}=0
        \\[10pt]
        {\rm F}=1
    \end{cases}
    \qquad
    {\rm Car\ Type}\begin{cases}
        {\rm Family}=0
        \\[10pt]
        {\rm Sports}=1
        \\[10pt]
        {\rm Luxury}=2
    \end{cases}
    \qquad
    {\rm Shirt\ Size}\begin{cases}
        {\rm Small}=0
        \\[10pt]
        {\rm Medium}=1
        \\[10pt]
        {\rm Large}=2
        \\[10pt]
        {\rm Extra\ Large}=3
    \end{cases}
\end{align*}

The output of neural network is Class, denoting C0 or C1.



\subsubsection{Update the weight values after one iteration of the back propagation}
\textcolor{gray}{\it Using the neural network obtained above, show the weight values after one iteration of the back propagation algorithm, given the training instance ``(M, Family, Small, C0)''. Indicate your initial weight values and biases and the learning rate used.}

Initialize hyper-parameter $\eta=0.5$ and random parameters as follows:

\begin{table}[H]
    \centering
    \renewcommand{\arraystretch}{1.5}
    \setlength{\tabcolsep}{64pt}
    \caption{Random initialized parameters}
    \begin{tabular}{cc}
        \bottomrule[1.5pt]
        Parameter & Value \\
        \midrule[0.75pt]
        $\omega_{14}$ & 0.1 \\
        $\omega_{24}$ & 0.2 \\
        $\omega_{34}$ & $-0.1$ \\
        $\omega_{15}$ & $-0.1$ \\
        $\omega_{25}$ & 0.1 \\
        $\omega_{35}$ & 0.05 \\
        $\omega_{46}$ & 0.4 \\
        $\omega_{56}$ & $-0.3$ \\
        $b_4$ & 0.35 \\
        $b_5$ & $-0.35$ \\
        $b_6$ & 0.1 \\
        \toprule[1.5pt]
    \end{tabular}
\end{table}

\paragraph*{Feed forward}~{}

\begin{align*}
    z_4&=[\omega_{14},\ \omega_{24},\ \omega_{34}]\left[\begin{matrix}
        x_{1} \\
        x_{2} \\
        x_{3} \\
    \end{matrix}\right]+b_4=[0.1,\ 0.2,\ -0.1]\left[\begin{matrix}
        0 \\
        0 \\
        0 \\
    \end{matrix}\right]+0.35=0.35
    \\
    a_4&=\sigma(z_4)=\dfrac{1}{1+\e^{-z_4}}\approx0.5866175789173301
    \\
    z_5&=[\omega_{15},\ \omega_{25},\ \omega_{35}]\left[\begin{matrix}
        x_{1} \\
        x_{2} \\
        x_{3} \\
    \end{matrix}\right]+b_5=[-0.1,\ 0.1,\ 0.05]\left[\begin{matrix}
        0 \\
        0 \\
        0 \\
    \end{matrix}\right]-0.35=-0.35
    \\
    a_5&=\sigma(z_5)=\dfrac{1}{1+\e^{-z_5}}=\approx0.41338242108267
    \\
    z_6&=[\omega_{46},\ \omega_{56}]\left[\begin{matrix}
        a_4 \\
        a_5 \\
    \end{matrix}\right]+b_6=[0.4,\ -0.3]\left[\begin{matrix}
        0.5866175789173301 \\
        0.41338242108267 \\
    \end{matrix}\right]+0.1=0.21063230524213108
    \\
    a_6&=\sigma(z_6)=\dfrac{1}{1+\e^{-z_6}}\approx0.5524642506473584
\end{align*}
\[ E=\dfrac{1}{2}(y-\hat{y})^2=\dfrac{1}{2}(0-a_6)^2=\dfrac{1}{2}(0-0.5524642506473584)^2\approx0.1526083741216736 \]

\paragraph*{Back-propagation}~{}

\begin{align*}
    \mathcal{L}_6&=\piff{E}{z_6}=\piff{E}{a_6}\cdot\piff{a_6}{z_6}=\eval{\diff{\left[\dfrac{1}{2}(y-\hat{y})^2\right]}{\hat{y}}}{\displaystyle\hat{y}=a_6}\cdot\eval{\diff{\left[\dfrac{1}{1+\e^{-x}}\right]}{x}}{\displaystyle x=z_6}=(a_6-y)a_6(1-a_6)
    \\
    &=(0.5524642506473584-0)\times0.5524642506473584\times(1-0.5524642506473584)
    \\
    &\approx0.13659540614006296
\end{align*}

According to:
\begin{align*}
    z_6=[\omega_{46},\ \omega_{56}]\left[\begin{matrix}
        a_4 \\
        a_5 \\
    \end{matrix}\right]+b_6\simeq\begin{cases}
        f(\bm{x})=\sum\bm{\omega}\bm{x}+b\implies f'(x)=\omega
        \\[10pt]
        f(\bm{\omega})=\sum\bm{x}\bm{\omega}+b\implies f'(\omega)=x
        \\[10pt]
        f(b)=b+\sum\bm{\omega}\bm{x}\implies f'(b)\equiv1
    \end{cases}
\end{align*}

Then:
\begin{align*}
    \mathcal{L}_5&=\piff{E}{z_5}=\piff{E}{z_6}\cdot\piff{z_6}{a_5}\cdot\piff{a_5}{z_5}=\mathcal{L}_6\times\omega_{56}\times a_5(1-a_5)
    \\
    &=0.13659540614006296\times(-0.3)\times0.41338242108267\times(1-0.41338242108267)
    \\
    &\approx-0.009937209048301705
\end{align*}
\begin{align*}
    \mathcal{L}_4&=\piff{E}{z_4}=\piff{E}{z_6}\cdot\piff{z_6}{a_4}\cdot\piff{a_4}{z_4}=\mathcal{L}_6\times\omega_{46}\times a_4(1-a_4)
    \\
    &=0.13659540614006296\times0.4\times0.5866175789173301\times(1-0.5866175789173301)
    \\
    &\approx0.013249612064402273
\end{align*}

Therefore:
\begin{align*}
    \omega'_{46}&=\omega_{46}-\eta\piff{E}{\omega_{46}}=\omega_{46}-\eta\cdot\piff{E}{z_6}\cdot\piff{z_6}{\omega_{46}}=\omega_{46}-\eta\mathcal{L}_6a_4
    \\
    &=0.4-0.5\times0.13659540614006296\times0.5866175789173301
    \\
    &=0.35993536677944343
\end{align*}
\begin{align*}
    \omega'_{56}&=\omega_{56}-\eta\piff{E}{\omega_{56}}=\omega_{56}-\eta\cdot\piff{E}{z_6}\cdot\piff{z_6}{\omega_{56}}=\omega_{56}-\eta\mathcal{L}_6a_5
    \\
    &=-0.3-0.5\times0.13659540614006296\times0.41338242108267
    \\
    &=-0.3282330698494749
\end{align*}
\begin{align*}
    \omega'_{14}&=\omega_{14}-\eta\piff{E}{\omega_{14}}=\omega_{14}-\eta\cdot\piff{E}{z_4}\cdot\piff{z_4}{\omega_{14}}=\omega_{14}-\eta\mathcal{L}_4x_1=\omega_{14}
    \\
    \omega'_{24}&=\omega_{24}-\eta\piff{E}{\omega_{24}}=\omega_{24}-\eta\cdot\piff{E}{z_4}\cdot\piff{z_4}{\omega_{24}}=\omega_{24}-\eta\mathcal{L}_4x_2=\omega_{24}
    \\
    \omega'_{34}&=\omega_{34}-\eta\piff{E}{\omega_{34}}=\omega_{34}-\eta\cdot\piff{E}{z_4}\cdot\piff{z_4}{\omega_{34}}=\omega_{34}-\eta\mathcal{L}_4x_3=\omega_{34}
\end{align*}
\begin{align*}
    \omega'_{15}&=\omega_{15}-\eta\piff{E}{\omega_{15}}=\omega_{15}-\eta\cdot\piff{E}{z_5}\cdot\piff{z_5}{\omega_{15}}=\omega_{15}-\eta\mathcal{L}_5x_1=\omega_{15}
    \\
    \omega'_{25}&=\omega_{25}-\eta\piff{E}{\omega_{25}}=\omega_{25}-\eta\cdot\piff{E}{z_5}\cdot\piff{z_5}{\omega_{25}}=\omega_{25}-\eta\mathcal{L}_5x_2=\omega_{25}
    \\
    \omega'_{35}&=\omega_{35}-\eta\piff{E}{\omega_{35}}=\omega_{35}-\eta\cdot\piff{E}{z_5}\cdot\piff{z_5}{\omega_{35}}=\omega_{35}-\eta\mathcal{L}_5x_3=\omega_{35}
\end{align*}

And these biases:
\begin{align*}
    b'_6&=b_6-\eta\piff{E}{b_6}=b_6-\eta\cdot\piff{E}{z_6}\cdot\piff{z_6}{b_6}=b_6-\eta\mathcal{L}_6
    \\
    &=0.1-0.5\times0.13659540614006296
    \\
    &\approx0.031702296929968524
\end{align*}
\begin{align*}
    b'_5&=b_5-\eta\piff{E}{b_5}=b_5-\eta\cdot\piff{E}{z_5}\cdot\piff{z_5}{b_5}=b_5-\eta\mathcal{L}_5
    \\
    &=-0.35-0.5\times(-0.009937209048301705)
    \\
    &\approx-0.3450313954758491
\end{align*}
\begin{align*}
    b'_4&=b_4-\eta\piff{E}{b_4}=b_4-\eta\cdot\piff{E}{z_4}\cdot\piff{z_4}{b_4}=b_4-\eta\mathcal{L}_4
    \\
    &=0.35-0.5\times0.013249612064402273
    \\
    &\approx0.34337519396779886
\end{align*}

Summarize:
\begin{table}[H]
    \centering
    \renewcommand{\arraystretch}{1.5}
    \setlength{\tabcolsep}{48pt}
    \caption{Parameters of one iteration of BP algorithm}
    \begin{tabular}{cc}
        \bottomrule[1.5pt]
        Parameter & Update \\
        \midrule[0.75pt]
        $\omega_{14}$ & $0.1\to\color{red}0.1$ \\
        $\omega_{24}$ & $0.2\to\color{red}0.2$ \\
        $\omega_{34}$ & $-0.1\to\color{red}-0.1$ \\
        $\omega_{15}$ & $-0.1\to\color{red}-0.1$ \\
        $\omega_{25}$ & $0.1\to\color{red}0.1$ \\
        $\omega_{35}$ & $0.05\to\color{red}0.05$ \\
        $\omega_{46}$ & $0.4\to\color{red}0.35993536677944343$ \\
        $\omega_{56}$ & $-0.3\to\color{red}-0.3282330698494749$ \\
        $b_4$ & $0.35\to\color{red}0.34337519396779886$ \\
        $b_5$ & $-0.35\to\color{red}-0.3450313954758491$ \\
        $b_6$ & $0.1\to\color{red}0.031702296929968524$ \\
        \toprule[1.5pt]
    \end{tabular}
\end{table}



\subsection{Na\"ve Bayesian Classifier}
\textcolor{gray}{\it Classify the unknown sample Z based on the training data set in Q1 (customer data overview): Z = (Gender = M, Car Type = Sports, Shirt Size = Small). What would a na\"ve Bayesian classifier classify Z?}

\[ p({\rm Class=C0})=p({\rm Class=C1})=\dfrac{10}{20}=\dfrac{1}{2} \]
\[ p({\rm Gender=M|Class=C0})=\dfrac{6}{6+4}=\dfrac{3}{5} \]
\[ p({\rm Gender=M|Class=C1})=\dfrac{4}{6+4}=\dfrac{2}{5} \]
\[ p({\rm Car\ Type=Sports|Class=C0})=\dfrac{8}{1+8+1}=\dfrac{4}{5} \]
\[ p({\rm Car\ Type=Sports|Class=C1})=\dfrac{0}{3+0+7}=0 \]
\[ p({\rm Shirt\ Size=Small|Class=C0})=\dfrac{3}{3+3+2+2}=\dfrac{3}{10} \]
\[ p({\rm Shirt\ Size=Small|Class=C1})=\dfrac{2}{3+3+2+2}=\dfrac{1}{5} \]
\begin{align*}
    p({\rm Z|Class=C0})&=p({\rm Gender=M|Class=C0})
    \\
    &\quad\times p({\rm Car\ Type=Sports|Class=C0})\times p({\rm Shirt\ Size=Small|Class=C0})
    \\
    &=\dfrac{3}{5}\times\dfrac{4}{5}\times\dfrac{3}{10}
\end{align*}
\begin{align*}
    p({\rm Z|Class=C1})&=p({\rm Gender=M|Class=C1})
    \\
    &\quad\times p({\rm Car\ Type=Sports|Class=C1})\times p({\rm Shirt\ Size=Small|Class=C1})
    \\
    &=\dfrac{2}{5}\times0\times\dfrac{1}{5}
\end{align*}
\begin{align*}
    p({\rm Z|Class=C0})p({\rm Class=C0})>p({\rm Z|Class=C1})p({\rm Class=C1})\implies \color{red}{\rm Z}\in{\rm C0}
\end{align*}



\subsection{Apriori Algorithm}
\textcolor{gray}{\it Consider the data set shown in table (min\_sup = 60\%, min\_conf=70\%).}

\begin{table}[H]
    \centering
    \renewcommand{\arraystretch}{1}
    \setlength{\tabcolsep}{48pt}
    \color{gray}{
        \caption{Example of market basket transactions}
        \begin{tabular}{|c|c|}
            \hline
            TID & Items-bought \\
            \hline
            T1 & \{A, D, B, C\} \\
            \hline
            T2 & \{D, A, C, E, B\} \\
            \hline
            T3 & \{A, B, E\} \\
            \hline
            T4 & \{A, B, D\} \\
            \hline
        \end{tabular}
    }
\end{table}



\subsubsection{Find all frequent itemsets}
\textcolor{gray}{\it Find all frequent itemsets using Apriori by treating each transaction ID as a market basket.}

\[ {\rm min\_count}=\lceil {\rm min\_sup}\times n\rceil =\lceil 60\%\times 4\rceil=3 \]

1st scan:
\begin{table}[H]
    \centering
    \renewcommand{\arraystretch}{1.2}
    \setlength{\tabcolsep}{30pt}
    \caption{Frequent itemset 1}
    \begin{tabular}{cccc}
        \bottomrule[1.5pt]
        Itemset & TID & Count & Pass \\
        \midrule[0.75pt]
        \{A\} & T1, T2, T3, T4 & 4 & $\checkmark$ \\
        \{B\} & T1, T2, T3, T4 & 4 & $\checkmark$ \\
        \{C\} & T1, T2 & 2 & \ding{55} \\
        \{D\} & T1, T2, T4 & 3 & $\checkmark$ \\
        \{E\} & T2, T3 & 2 & \ding{55} \\
        \toprule[1.5pt]
    \end{tabular}
\end{table}

2nd scan:
\begin{table}[H]
    \centering
    \renewcommand{\arraystretch}{1.2}
    \setlength{\tabcolsep}{30pt}
    \caption{Frequent itemset 2}
    \begin{tabular}{cccc}
        \bottomrule[1.5pt]
        Itemset & TID & Count & Pass \\
        \midrule[0.75pt]
        \{A, B\} & T1, T2, T3, T4 & 4 & $\checkmark$ \\
        \{A, D\} & T1, T2, T4 & 3 & $\checkmark$ \\
        \{B, D\} & T1, T2, T4 & 3 & $\checkmark$ \\
        \toprule[1.5pt]
    \end{tabular}
\end{table}

1rd scan:
\begin{table}[H]
    \centering
    \renewcommand{\arraystretch}{1.2}
    \setlength{\tabcolsep}{30pt}
    \caption{Frequent itemset 3}
    \begin{tabular}{cccc}
        \bottomrule[1.5pt]
        Itemset & TID & Count & Pass \\
        \midrule[0.75pt]
        \{A, B, D\} & T1, T2, T4 & 3 & $\checkmark$ \\
        \toprule[1.5pt]
    \end{tabular}
\end{table}

All the frequent itemsets are: \textcolor{red}{\{A\}, \{B\}, \{D\}, \{A, B\}, \{A, D\}, \{B, D\}, \{A, B, D\}}.



\subsubsection{Compute the confidence for the association rules}
\textcolor{gray}{\it Use the results in part (a) to compute the confidence for the association rules \{a, b\}$\to$\{c\} and \{c\}$\to$\{a, b\}. Is confidence a symmetric measure?}
\[ {\rm Suppor}(X\to Y)={\rm Suppor}(X\bigcup Y) \]
\[ {\rm Confidence}\left(\{A,\ B\}\to\{C\}\right)=\dfrac{{\rm Support}\left(\{A,\ B\}\bigcup\{C\}\right)}{{\rm Support}\left(\{A,\ B\}\right)}=\dfrac{\dfrac{2}{n}}{\dfrac{4}{n}}=50\% \]
\[ {\rm Confidence}\left(\{C\}\to\{A,\ B\}\right)=\dfrac{{\rm Support}\left(\{C\}\bigcup\{A,\ B\}\right)}{{\rm Support}\left(\{C\}\right)}=\dfrac{\dfrac{2}{n}}{\dfrac{2}{n}}=100\% \]
\[ {\rm Confidence}\left(\{A,\ B\}\to\{C\}\right)\neq{\rm Confidence}\left(\{C\}\to\{A,\ B\}\right) \]

Since here we have, the confidence is an \textcolor{red}{asymmetric} measure which depends on precondition.



\subsubsection{List all of the strong association rules}
\textcolor{gray}{\it List all of the strong association rules (with support s and confidence c) matching the following metarule, where $X$ is a variable representing customers, and ${\rm item}_i$ denotes variables representing items (e.g. ``A'', ``B'', etc.):}
\[ \color{gray}\forall x\in{\rm transactions},\quad {\rm buys}(X,\ {\rm item_1})\wedge{\rm buys}(X,\ {\rm item_2})=>{\rm buys}(X,\ {\rm item_3}),\quad [s,\ c] \]

Construct 3 association rules with ``\{A, B, D\}'':
\begin{align*}
    {\rm rule\ 1:}\quad \{A,\ B\}\to\{D\}
    \\
    {\rm rule\ 2:}\quad \{A,\ D\}\to\{B\}
    \\
    {\rm rule\ 3:}\quad \{B,\ D\}\to\{A\}
\end{align*}
\[ {\rm Suppor}(\{A,\ B\}\to\{D\})={\rm Suppor}(\{A,\ D\}\to\{B\})={\rm Suppor}(\{B,\ D\}\to\{A\})=75\% \]
\begin{align*}
    {\rm Confidence}\left(\{A,\ B\}\to\{D\}\right)&=\dfrac{{\rm Support}\left(\{A,\ B\}\bigcup\{D\}\right)}{{\rm Support}\left(\{A,\ B\}\right)}=\dfrac{3}{4}=75\%
    \\
    {\rm Confidence}\left(\{A,\ D\}\to\{B\}\right)&=\dfrac{{\rm Support}\left(\{A,\ D\}\bigcup\{B\}\right)}{{\rm Support}\left(\{A,\ D\}\right)}=\dfrac{3}{3}=100\%
    \\
    {\rm Confidence}\left(\{B,\ D\}\to\{A\}\right)&=\dfrac{{\rm Support}\left(\{B,\ D\}\bigcup\{A\}\right)}{{\rm Support}\left(\{B,\ D\}\right)}=\dfrac{3}{3}=100\%
\end{align*}

Since all the confidence values of association rule are higher than those of min\_conf, the 3 rules ($\{A,\ B\}\to\{D\}$, $\{A,\ D\}\to\{B\}$, $\{B,\ D\}\to\{A\}$) are \textcolor{red}{strong association rules}.



\section{Lab Assignment}
\subsection{C5.0 Algorithm for Transactions}
\textcolor{gray}{\it Assume this supermarket would like to promote milk. Use the data in ``transactions'' as training data to build a decision tree (C5.0 algorithm) model to predict whether the customer would buy pasta or not.}



\subsubsection{A figure showing decision tree}
\textcolor{gray}{\it Build a decision tree using data set ``transactions'' that predicts milk as a function of the other fields. Set the ``type'' of each field to ``Flag'', set the ``direction'' of ``pasta'' as ``out'', set the ``type'' of COD as ``Typeless'', select ``Expert'' and set the ``pruning severity'' to 65, and set the ``minimum records per child branch'' to be 95.}

\textcolor{gray}{\it\bfseries Hand-in: a figure showing your tree.}

\begin{figure}[H]
    \centering
    \includegraphics*[scale=0.28]{./figs/DecisionTree.png}
    \caption{C5.0 decision tree}
\end{figure}



\subsubsection{Prediction for each of the 20 customers}
\textcolor{gray}{\it Use the model (the full tree generated by Clementine in step 1 above) to make a prediction for each of the 20 customers in the ``rollout'' data to determine whether the customer would buy pasta.}

\textcolor{gray}{\it\bfseries Hand-in: your prediction for each of the 20 customers.}

\begin{figure}[H]
    \centering
    \includegraphics*[scale=0.5]{./figs/20.png}
    \caption{Prediction for each of the 20 customersr}
\end{figure}



\subsubsection{Rules for positive prediction of pasta purchase}
\textcolor{gray}{\it\bfseries Hand-in: rules for positive (yes) prediction of pasta purchase identified from the decision tree (up to the fifth level. The root is considered as level 1).}

\begin{figure}[H]
    \centering
    \includegraphics*[scale=0.85]{./figs/5.png}
    \caption{Rules for positive prediction of pasta purchase}
\end{figure}

\begin{lstlisting}[
    basicstyle = \ttfamily,
    columns = flexible,
    breaklines = true,
    escapeinside = {!@}{@!}
]
Rules for 1 - contains !@\textcolor{red}{7}@! rule(s)
    Rule 1 for  1
        if tomato souce = 1
        and tunny = 1
        then 1
    Rule 2 for  1
        if tomato souce = 1
        and tunny = 0
        and rice = 1
        then 1
    Rule 3 for  1
        if tomato souce = 1
        and tunny = 0
        and rice = 0
        and brioches = 1
        then 1
    Rule 4 for  1
        if tomato souce = 1
        and tunny = 0
        and rice = 0
        and brioches = 0
        and frozen vegetables = 1
        then 1
    Rule 6 for  1
        if tomato souce = 0
        and rice = 1
        and coffee = 1
        then 1
    Rule 7 for  1
        if tomato souce = 0
        and rice = 1
        and coffee = 0
        and biscuits = 1
        then 1
    Rule 8 for  1
        if tomato souce = 0
        and rice = 1
        and coffee = 0
        and biscuits = 0
        and coke = 1
        then 1
\end{lstlisting}



\subsection{C5.0 Algorithm for Bank PEP}
\textcolor{gray}{\it Each record is a customer description where the ``pep'' field indicates whether or not that customer bought a PEP. For other existing customers in the database, we would like to see if PEP should be RECOMMENDED to the customers in the roll-out data.}

\textcolor{gray}{\it The firm decides to use decision tree to build the models for PEP recommendation. Develop a decision tree model using the estimation data. For building this model, you are expected to use the following steps.}

\textcolor{gray}{\it Using the ``bank-estimation-data'', estimate the decision tree that predicts pep as a function of the other variables. Select ``Expert'' and set ``pruning severity'' at 70. Set the ``Type'' of pep as ``Flag'' and the ``Direction'' as ``out''. Build decision trees using three options ``Minimum records per child branch'' values being (a) 56, (b) 15 and (c) 10, not selecting ``use global pruning''.}



\subsubsection{Confusion matrix for different strategies}
\textcolor{gray}{\it\bfseries Hand in: the confusion matrix for (a), (b) and (c) on the validation data.}

\begin{figure}[H]
	\centering
	\begin{minipage}{0.32\linewidth}
		\centering
		\includegraphics[width=0.95\linewidth]{./figs/56.png}
		\caption{Minimum records per child branch (a) 56}
	\end{minipage}
	\begin{minipage}{0.32\linewidth}
		\centering
		\includegraphics[width=0.95\linewidth]{./figs/15.png}
		\caption{Minimum records per child branch (b) 15}
	\end{minipage}
    \begin{minipage}{0.32\linewidth}
		\centering
		\includegraphics[width=1\linewidth]{./figs/10.png}
		\caption{Minimum records per child branch (c) 10}
	\end{minipage}
\end{figure}



\subsubsection{Select the optimal decision tree}
\textcolor{gray}{\it\bfseries Hand in: which of the three trees will you use to score the data in a holdout data list and why? 2-3 lines.}

\paragraph*{Confusion matrix for (a)}~{}
\begin{align*}
    {\rm precision}&=0.7352941176470589 \\
    {\rm recall}&=0.78125 \\
    {\rm F1}&=0.7575757575757576 \\
\end{align*}

\paragraph*{Confusion matrix for (b)}~{}
\begin{align*}
    {\rm precision}&=0.8780487804878049 \\
    {\rm recall}&=0.75 \\
    {\rm F1}&=0.8089887640449439 \\
\end{align*}

\paragraph*{Confusion matrix for (c)}~{}
\begin{align*}
    {\rm precision}&=0.8809523809523809 \\
    {\rm recall}&=0.7708333333333334 \\
    {\rm F1}&=0.8222222222222222 \\
\end{align*}

\textcolor{red}{The optimal decision tree is tree (c)}. Because the tree (c) F1 score is the highest, the tree (a) underfits the data, resulting in a lower metrics. The tree (b) and tree (c) gradually refine the branches of the tree, which helps to achieve consistent performance improvements without overfitting the data.


\subsubsection{Predict the appendix data with the optimal decision tree}
\textcolor{gray}{\it\bfseries Hand in: for the following data appendix, using the rules from the best decision tree, fill in the recommendation.}

\begin{figure}[H]
    \centering
    \includegraphics*[scale=0.75]{./figs/appendix.png}
    \caption{Prediction of the recommendation with the optimal decision tree (c)}
\end{figure}



\end{document}